%% 
%% Copyright 2007, 2008, 2009 Elsevier Ltd
%% 
%% This file is part of the 'Elsarticle Bundle'.
%% ---------------------------------------------
%% 
%% It may be distributed under the conditions of the LaTeX Project Public
%% License, either version 1.2 of this license or (at your option) any
%% later version.  The latest version of this license is in
%%    http://www.latex-project.org/lppl.txt
%% and version 1.2 or later is part of all distributions of LaTeX
%% version 1999/12/01 or later.
%% 
%% The list of all files belonging to the 'Elsarticle Bundle' is
%% given in the file `manifest.txt'.
%% 
%% Template article for Elsevier's document class `elsarticle'
%% with harvard style bibliographic references
%% SP 2008/03/01

\documentclass[preprint,12pt,authoryear]{elsarticle}

%% Use the option review to obtain double line spacing
%% \documentclass[authoryear,preprint,review,12pt]{elsarticle}

%% Use the options 1p,twocolumn; 3p; 3p,twocolumn; 5p; or 5p,twocolumn
%% for a journal layout:
%% \documentclass[final,1p,times,authoryear]{elsarticle}
%% \documentclass[final,1p,times,twocolumn,authoryear]{elsarticle}
%% \documentclass[final,3p,times,authoryear]{elsarticle}
%% \documentclass[final,3p,times,twocolumn,authoryear]{elsarticle}
%% \documentclass[final,5p,times,authoryear]{elsarticle}
%% \documentclass[final,5p,times,twocolumn,authoryear]{elsarticle}


\usepackage[utf8]{inputenc}



%% For including figures, graphicx.sty has been loaded in
%% elsarticle.cls. If you prefer to use the old commands
%% please give \usepackage{epsfig}

%% The amssymb package provides various useful mathematical symbols
\usepackage{amssymb}
\usepackage{color}
\usepackage{url}
\usepackage{hyperref}
\usepackage{graphicx,array}
\usepackage{amsmath}
\usepackage{siunitx}

%% The lineno packages adds line numbers. Start line numbering with
%% \begin{linenumbers}, end it with \end{linenumbers}. Or switch it on
%% for the whole article with \linenumbers.
%% \usepackage{lineno}

\newcommand{\nati}[1]{{\color[rgb]{.1,.6,.1}{#1}}}

\newcommand{\todo}[1]{{\color[rgb]{.6,.1,.6}{#1}}}

\newcommand{\assign}[1]{{\color[rgb]{.8,.5,.8}{Assigned: #1 }}}




\journal{Astronomy and Computing}

\begin{document}

\begin{frontmatter}

%% Title, authors and addresses

%% use the tnoteref command within \title for footnotes;
%% use the tnotetext command for theassociated footnote;
%% use the fnref command within \author or \address for footnotes;
%% use the fntext command for theassociated footnote;
%% use the corref command within \author for corresponding author footnotes;
%% use the cortext command for theassociated footnote;
%% use the ead command for the email address,
%% and the form \ead[url] for the home page:
%% \title{Title\tnoteref{label1}}
%% \tnotetext[label1]{}
%% \author{Name\corref{cor1}\fnref{label2}}
%% \ead{email address}
%% \ead[url]{home page}
%% \fntext[label2]{}
%% \cortext[cor1]{}
%% \address{Address\fnref{label3}}
%% \fntext[label3]{}

\title{Efficient spherical Convolutional Neural Networks with Healpix sampling for cosmological applications}

%% use optional labels to link authors explicitly to addresses:
%% \author[label1,label2]{}
%% \address[label1]{}
%% \address[label2]{}

\author{}

\address{}

\begin{abstract}
%% Text of abstract

	\todo{(Michaël) I would not invent a new term, SCNN, but rather say that convolutional neural networks on graphs (or GCNs) can be efficiently applied to (spherical?) cosmological applications.
	Arguments: the method is not new and hence don't deserve a new name. Emphasize that it is generic to the data structure, and that the sphere is simply a particular graph.
	Ideas: Graph Convolutional Networks for efficient spherical ???, Efficient spherical ??? with GCNs and Healpix sampling
}

\end{abstract}

\begin{keyword}
%% keywords here, in the form: keyword \sep keyword

%% PACS codes here, in the form: \PACS code \sep code

%% MSC codes here, in the form: \MSC code \sep code
%% or \MSC[2008] code \sep code (2000 is the default)

\end{keyword}

\end{frontmatter}

%% \linenumbers

%% main text
\section{Introduction}
\label{sec:intro}

%\subsection{Motivation}
\assign{Tomek}

a) Cosmology has a lot of spherical data. Our method is simple and easy to use. Moreover, it is based on the widely used healpix sampling.
b) It is the most efficient spherical convolution\footnote{\todo{provably? cannot be faster than O(n) without approximations, e.g. sketching}}, requiring only $O(n)$ operations, where $n$ is the number of points.

\subsection{Potential applications}
	\assign{Tomek, Nathanaël, Michaël}

The analysis of spherical cosmological data, such as the cosmic microwave background \cite{...}, as done in \cite{he2018analysis}, is the target application of our method.

While our method was developed with cosmology in mind, it can easily target any problem where data live on a sphere. Examples include, but are certainly not limited to, (i) efficient compression and decompression of \ang{360} videos (see \cite{su2017learning}), (ii) \todo{data analysis on planets? (climate, forecasting, temperature, wind)}, (iii) \todo{particle physics? (jets on detectors, but they are usually cylindrical)}, (iv) \todo{applications in Cohen's papers?}.

Finally, not that those neural networks are not restricted to the sphere and can be applied to any problem where we have data on a graph, such as social, biological or infrastructure networks [some citations, e.g. brain Alzeihmer, particle physics, computer graphics].
% the convolution is not restricted to the sphere, the coarsening/pooling is


\todo{
\subsection{Healpix sampling}
Should we have this subsection to explain the advantages of Healpix?
\begin{itemize}
	\item Define the gedesic grid?
	\item reference the "cubed sphere"~\cite{ronchi1996cubed}?
\end{itemize}
}

\begin{figure}[!ht]
\centering
\includegraphics[width=0.95\textwidth]{figures/healpix-3layers.jpg}
\caption{3 smaller sclales of the healpix sampling.}
\label{fig:healpix_sampling}
\end{figure}

\section{Related work}
\label{sec:related}



\subsection{Spherical convolutions} 
The major challenge to generalize CNN to a
spherical domain is to generalize convolution. In the context of CNN, we found
two particular approaches to address this issue.

The first approach, leveraging the continuous spherical convolution, has been
proposed in~\cite{cohen2017convolutional,cohen2018spherical}. These
contributions propose to use a convolution that is rotational equivariant on the
sphere, i.e. a rotation of the input implies the same rotation of the output.
The convolution is performed by a spherical Fourier transform (i.e. a projection
on the spherical harmonics), a  multiplication in the spectral domain and an
inverse spherical Fourier tranform. Hence the main computational cost of a
convolution comes from the two Fourier transforms. Fortunately, the Fourier
transform can be accelerated for special samplings set (theoretically including
HealPix). The main advantage of this approach is that it provides a
mathematically well defined rotational equivariant network. Nevertheless, even
with these Fourier transform acceleration, the convolutions remain expensive in
comparison to a traditional 2D convolution, limiting the practical use of this
approach. While a comparison is theoretically possible, the experiments
of~\cite{cohen2018spherical} are done using the geodesic grid sampling set and
do not provide any code for the Healpix one. Hence, a practical comparison is
not possible.

A second direction has been followed in~\cite{boomsma2017spherical}. The idea is
to use the traditional 2D convolution on an irregular grid defined on the
sphere. In the paper, two grids are used: the geodesic grid and the “cubed
sphere” grid defined by Ronchi et. al~\cite{ronchi1996cubed}. Because this
approach is based on the traditional 2D convolution, it is very likely to be the
most efficient one. However, it suffers that it can only be used using very
specific grid sampling sets which are NOT including HealPix. Furthermore, the
grid requirement makes it impossible for the convolution to capture the
spherical properties of the domain, i.e. the "cubed sphere” sampling is by
definition adapted to a cube and not to a sphere. \nati{Michael: Under some very
specific hypothesis, this second approach is a particular case of our method,
i.e. 1) our method with a stupid sampling, 2) assumption that graph convolution
on a grid == 2d convolution. Do you think we should mention that? I think we
should not.}

In an attempt to get the best of the two approaches, we follow a third
direction. We use a graph to adapt to the particular structure of the sphere.
While the convolution remains efficient, it still captures well the spherical
structure and is particularly adapted to the HealPix sampling.

\subsection{Convolutions on graphs}
\assign{Michaël}

\todo{other approaches? GNNs, Kipf first order approx, message passing}

Previous work: [Bruna] which needed the full eigendecomposition of the Laplacian, costing $O(n^3)$ operations.

In this work, we are using the graph CNN formulation introduced in~\cite{defferrard2016convolutional}.

Spatial definitions of graph convolutions, e.g. [Niepert] needs to define an orientation in order to match the edges with the filters. Most often the orientation is not given by the application, and one has to define it (for example by ordering by degree or any other measure, or by using a graph coloring). There is no good default good orientation on general graphs and the choice of an orientation is highly application dependent.

\todo{Cool to have a global illustration of the network (CNN like)}

\subsection{Convolutions on manifolds}

Related to this, convolutional neural networks have been defined on manifolds and have achieved impressive results on shapes [Bronstein]. They however too depend on an orientation, which spheres do not possess.

\subsection{Convolutions on point clouds?}

PointNet and co. Related but we are loosing the structure. Also coarsening.

\section{Method}



The gist of our method is to define the convolution on a sphere using a graph. The graph is here seen as a discrete approximation of the sphere $S^2$, a 2-dimensional manifold embedded in $\mathbb{R}^3$.

As presented in~\cite{cohen2018spherical}, the most mathematical approach to extend the convolution on a sphere is to use a spherical Fourier transform. The convolution is then simply defined as the product in the spectral domain. This approach requires one Fourier and one inverse Fourier transform per convolution which is, even with accelerated algorithms, rather expensive. For 2-dimensional images, an efficient convolution can be achieved when the convolution kernel is localized (for example a 5x5 pixel patch) by doing the computation directly in the signal domain. Unfortunately, this approach cannot be directly extended to the spherical case. Hence, the main idea of this contribution is to leverage graph signal processing~\cite{shuman2013emerging} to define a spherical convolution that can be computed directly in the signal domain. 

In this section, we follow classical constructions presented in~\cite{cohen2018spherical,...} and \nati{show} that the resulting spherical convolution is close to one corresponding to spherical approach.

\subsection{Graph creation}
\assign{Nathanaël}

\begin{figure}[!ht]
\centering
\includegraphics[width=0.45\textwidth]{figures/healpix_graph_4.pdf}
\includegraphics[width=0.45\textwidth]{figures/half_graph_4.pdf}
\caption{Left: Healpix graph for the full sphere using a nside parameter of $4$. Right: Top view of a Healpix graph for half the sphere using a nside parameter of $4$.}
\label{fig:healpix_graph_4}
\end{figure}

From the ubiquitous HealPix sampling~\citep{gorski2005healpix}, we create a weighted undirected graph where each pixel is a node (vertex) that is connected to his $8$ or $7$ closest neighbors.\footnote{For some pixels, the $8^{th}$ nearest neighbor is not well defined.} Given the set of nearest neighbors, we define the weight matrix $W$ using the following scheme
\begin{equation}
W[i,j]=\begin{cases}
e^{-\frac{\|x_i-x_j\|_2^2}{\sigma^2}} & \text{if pixels $i$ and $j$ are connected, and}\\
0 & \text{otherwise.}\\
\end{cases}
\end{equation}
Here $x_i$ is a 3-dimensional vector encoding the coordinate of the pixels $i$ on the sphere and $\sigma$ is the mean of $\|x_i-x_j\|_2$ over all connected pixels $i$ and $j$. The degree of a node (or a pixel) is defined as $d_i = \sum_j W[i,j]$. The degree matrix $D$ is the diagonal matrix where $D_{ii}=d_i$.

Figure Graph (show also the triangle, separate plot?) + sampling

\subsection{Coarsening}
\assign{Nathanaël}

How we get coarser or finer sampling of the sphere with Healpix. The $2^k \cdot 12$ points.

\begin{figure}[!ht]
\centering
\includegraphics[width=0.95\textwidth]{figures/pooling.pdf}
\caption{Pooling, 1/12 of the sphere, 2 levels, molview projection.}
\label{fig:pooling}
\end{figure}

\subsection{Graph Fourier basis and spherical harmonics}
\assign{Nathanaël}
% TODO: basis or transform?

The graph normalized graph Laplacian defined as $L = I - D^{-1/2} W D^{-1/2}$ is a second order differential operator that can be used to define the graph Fourier basis. 

\begin{figure}[!ht]
\centering
\includegraphics[width=0.95\textwidth]{figures/eigenvectors.pdf}
\caption{12 first graph Fourier eigenvectors.}
\label{fig:graph_harmonics}
\end{figure}
 
\begin{itemize}
	\item We build a graph using the healpix sampling
	\item Define Fourier transform and show that the harmonics are visually close to the spherical harmonics
	\item Define spherical convolution using the graph Fourier transform and show heat diffusion example
	\item Show the limits of the approach and explain why we cannot have a perfect spherical convolution with this technique
\end{itemize}

\subsection{Convolutions on graphs}
\assign{Nathanaël, Michaël}

With the Fourier basis. Because we don't have an FFT, that costs $O(n^3)$ for the eigendecomposition, plus $O(n^2)$ for the transform (to be done for each forward and backward pass).

Example on the heat equation
\begin{figure}[!ht]
\centering
\includegraphics[width=0.95\textwidth]{figures/gaussian_filters_sphere.pdf}
\caption{Top: Diffusion of a unit of heat for different time using the graph. Bottom: gaussian filtering on the sphere. Relative difference of the two diffusions: $13.5$\%, $6.1$\%, $5.2$\%, $3.8$\% }
\label{fig:gaussian_filters_comparizon}
\end{figure}

Figure filters 3 representations for both graph + spherical Fourier
\begin{figure}[!ht]
\centering
\includegraphics[width=0.95\textwidth]{figures/gaussian_filters_spectral.pdf}
\includegraphics[width=0.95\textwidth]{figures/gaussian_filters_gnomonic.pdf}
\includegraphics[width=0.95\textwidth]{figures/gaussian_filters_section.pdf}
\caption{3 Different effet visualization of filters. Top: Graph spectral domain. Middle: gnomonic projection on the sphere. Bottom: section of filter.}
\label{fig:gaussian_filters_visualization}
\end{figure}


\begin{figure}[!ht]
\centering
\includegraphics[width=0.45\textwidth]{figures/index_plotting_order20_nside16.pdf}

\caption{Indexes selected for the section ploting of the filter.}
\label{fig:index_section}
\end{figure}


\subsection{Efficient convolutions}
\assign{Michaël}

\begin{itemize}
	\item We need efficient convolution, hence the Chebysheff trick
	\item Explain the different between a polynomial filter convolution and a traditional patch restricted convolution
	\item Define spherical CNN using graph CNN
\end{itemize}

\section{Experiments}

What do we want to show? The graph CNN is able to discriminate ??? using higher order statistics.

\subsection{Data}
\assign{Tomek}

Our data comes from a simulation with software ??? using the following parameters ???. All the data to reproduce our experiments are available online.\footnote{\url{https://doi.org/10.5281/zenodo.????} \todo{correct DOI}}

\begin{figure}[!ht]
\centering
\includegraphics[width=0.45\textwidth]{figures/smooth_map_class_1.pdf}
\includegraphics[width=0.45\textwidth]{figures/smooth_map_class_2.pdf}
\caption{One sample of each class }
\label{fig:map_sample}
\end{figure}

\begin{figure}[!ht]
\centering
\includegraphics[width=0.45\textwidth]{figures/part_sphere.pdf}
\caption{3 subparts of the sphere with different sizes. Blue: order 1. Green: order 2. Yellow: order 3.}
\label{fig:subpart_sphere}
\end{figure}

\subsection{Results}

\begin{figure}[!ht]
\centering
\includegraphics[width=0.32\textwidth]{figures/result_order1.pdf}
\includegraphics[width=0.32\textwidth]{figures/result_order2.pdf}
\includegraphics[width=0.32\textwidth]{figures/result_order4.pdf}
\caption{Classification errors for the 3 different problems.}
\label{fig:results}
\end{figure}

\assign{Nathanaël, Tomek}

\begin{itemize}
	\item With the entire sphere / different PSD (The PSD features + linear SVM is equivalently good)
	\item Without the entire sphere / same PSD (1 sample: the histogram features + kernelized SVM is equivalently good)
	\item ??
\end{itemize}


Figure: Result curves 

Figure: Filters 1D and Gnonomic

\subsection{Interpretation}
\assign{Nathanaël, Tomek, Michaël}

Show the learned filters and feature maps and try to interpret them.

\section{Conclusion}
\assign{Nathanaël, Tomek, Michaël}

\section*{Thanks}

%% The Appendices part is started with the command \appendix;
%% appendix sections are then done as normal sections
%% \appendix

%% \section{}
%% \label{}

%% If you have bibdatabase file and want bibtex to generate the
%% bibitems, please use
%%
\section*{Bibliography}
\bibliographystyle{elsarticle-harv} 
\bibliography{biblio}

\end{document}

\endinput
%%
%% End of file `elsarticle-template-harv.tex'.
